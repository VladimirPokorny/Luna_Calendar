\documentclass[border=40mm, 10pt, varwidth=true, varwidth=610mm]{standalone}

\usepackage[pass, paperwidth=594mm]{geometry}


\usepackage{tikz}
\usetikzlibrary{calendar, fpu}
\usepackage{lmodern}

\tikzset{
    moon colour/.style={
        moon fill/.style={
            fill=#1
        }
    },
    sky colour/.style={
        sky draw/.style={
            draw=#1
        },
        sky fill/.style={
            fill=#1
        }
    },
    southern hemisphere/.style={
        rotate=180
    }
}

\makeatletter
\pgfcalendardatetojulian{2022-01-02}{\c@pgf@counta}
\def\synodicmonth{29.530588853}

\newcommand{\numberDay}[1]{%
    \tikzset{red/.style={rectangle, draw=white, fill= white , minimum width = 5pt, minimum height=\moonsize em}}
    \begin{tikzpicture}
        \node[red](r1){\Huge{#1.}};
    \end{tikzpicture}
    }

\newcommand{\moon}[3][]{%
    \edef\checkfordate{\noexpand\in@{-}{#2}}%
    \checkfordate%
    \ifin@%
        \pgfcalendardatetojulian{#2}{\c@pgf@countb}%
        \pgfkeys{/pgf/fpu=true,/pgf/fpu/output format=fixed}%
        \pgfmathsetmacro\dayssincenewmoon{\the\c@pgf@countb-\the\c@pgf@counta-(7/24+11/(24*60))}%
        \pgfmathsetmacro\lunarage{mod(\dayssincenewmoon,\synodicmonth)}
        \pgfkeys{/pgf/fpu=false}%%
    \else%
        \def\lunarage{#2}%
    \fi%
    \pgfmathsetmacro\leftside{ifthenelse(\lunarage<=\synodicmonth/2,cos(360*(\lunarage/\synodicmonth)),1)}%
    \pgfmathsetmacro\rightside{ifthenelse(\lunarage<=\synodicmonth/2,-1,-cos(360*(\lunarage/\synodicmonth))}%
    \tikz [moon colour=white,sky colour=black,#1]{
        \draw [moon fill, sky draw] (0,0) circle [radius=1ex];
        \draw [sky draw, sky fill] (0,1ex)
            arc (90:-90:\rightside ex and 1ex)
            arc (-90:90:\leftside ex and 1ex)
            -- cycle;
        \draw [white] (0,0) circle [radius=1.2ex];
        \node at (-1ex,1ex) {#3};
    }%
}

\newcommand{\moonsize}{9}
\newcommand{\skycolour}{black!70}

\begin{document}

\begin{tabular}{c}
    \fontsize{140}{120}\selectfont LunaCalendar 2024 \vspace{1cm} \\
    \multicolumn{1}{c}{\large{By Vladimír Pokorný $\copyright$}} \\
    \vspace{1cm}\\ \hline
    \vspace{1cm}\\
    \begin{tabular}{cccccccccccccc} 
	& \Huge{January}	& \Huge{February}	& \Huge{March}	& \Huge{April}	& \Huge{May}	& \Huge{June}	& \Huge{July}	& \Huge{August}	& \Huge{September}	& \Huge{October}	& \Huge{November}	& \Huge{December}\\ &&&&&&&&&&&& \\ 
	 \numberDay{1} & \moon[scale=\moonsize, sky colour=\skycolour]{19.027805}{Mon} &	 \moon[scale=\moonsize, sky colour=\skycolour]{20.551263}{Thu} &	 \moon[scale=\moonsize, sky colour=\skycolour]{20.119294}{Fri} &	 \moon[scale=\moonsize, sky colour=\skycolour]{21.728780}{Mon} &	 \moon[scale=\moonsize, sky colour=\skycolour]{22.352755}{Wed} &	 \moon[scale=\moonsize, sky colour=\skycolour]{23.977227}{Sat} &	 \moon[scale=\moonsize, sky colour=\skycolour]{24.557221}{Mon} &	 \moon[scale=\moonsize, sky colour=\skycolour]{26.060924}{Thu} &	 \moon[scale=\moonsize, sky colour=\skycolour]{27.456150}{Sun} &	 \moon[scale=\moonsize, sky colour=\skycolour]{27.756818}{Tue} &	 \moon[scale=\moonsize, sky colour=\skycolour]{29.001781}{Fri} &	 \moon[scale=\moonsize, sky colour=\skycolour]{29.267536}{Sun}	 & \numberDay{1}\\ 
	 \numberDay{2} & \moon[scale=\moonsize, sky colour=\skycolour]{20.028244}{Tue} &	 \moon[scale=\moonsize, sky colour=\skycolour]{21.553674}{Fri} &	 \moon[scale=\moonsize, sky colour=\skycolour]{21.123137}{Sat} &	 \moon[scale=\moonsize, sky colour=\skycolour]{22.733591}{Tue} &	 \moon[scale=\moonsize, sky colour=\skycolour]{23.358027}{Thu} &	 \moon[scale=\moonsize, sky colour=\skycolour]{24.982148}{Sun} &	 \moon[scale=\moonsize, sky colour=\skycolour]{25.560629}{Tue} &	 \moon[scale=\moonsize, sky colour=\skycolour]{27.061592}{Fri} &	 \moon[scale=\moonsize, sky colour=\skycolour]{28.453371}{Mon} &	 \moon[scale=\moonsize, sky colour=\skycolour]{28.750981}{Wed} &	 \moon[scale=\moonsize, sky colour=\skycolour]{0.464122}{Sat} &	 \moon[scale=\moonsize, sky colour=\skycolour]{0.731688}{Mon}	 & \numberDay{2}\\ 
	 \numberDay{3} & \moon[scale=\moonsize, sky colour=\skycolour]{21.028684}{Wed} &	 \moon[scale=\moonsize, sky colour=\skycolour]{22.556086}{Sat} &	 \moon[scale=\moonsize, sky colour=\skycolour]{22.126980}{Sun} &	 \moon[scale=\moonsize, sky colour=\skycolour]{23.738403}{Wed} &	 \moon[scale=\moonsize, sky colour=\skycolour]{24.363298}{Fri} &	 \moon[scale=\moonsize, sky colour=\skycolour]{25.987069}{Mon} &	 \moon[scale=\moonsize, sky colour=\skycolour]{26.564036}{Wed} &	 \moon[scale=\moonsize, sky colour=\skycolour]{28.062260}{Sat} &	 \moon[scale=\moonsize, sky colour=\skycolour]{29.450592}{Tue} &	 \moon[scale=\moonsize, sky colour=\skycolour]{0.214234}{Thu} &	 \moon[scale=\moonsize, sky colour=\skycolour]{1.457343}{Sun} &	 \moon[scale=\moonsize, sky colour=\skycolour]{1.726976}{Tue}	 & \numberDay{3}\\ 
	 \numberDay{4} & \moon[scale=\moonsize, sky colour=\skycolour]{22.029124}{Thu} &	 \moon[scale=\moonsize, sky colour=\skycolour]{23.558497}{Sun} &	 \moon[scale=\moonsize, sky colour=\skycolour]{23.130822}{Mon} &	 \moon[scale=\moonsize, sky colour=\skycolour]{24.743214}{Thu} &	 \moon[scale=\moonsize, sky colour=\skycolour]{25.368569}{Sat} &	 \moon[scale=\moonsize, sky colour=\skycolour]{26.991991}{Tue} &	 \moon[scale=\moonsize, sky colour=\skycolour]{27.567443}{Thu} &	 \moon[scale=\moonsize, sky colour=\skycolour]{29.062928}{Sun} &	 \moon[scale=\moonsize, sky colour=\skycolour]{0.914412}{Wed} &	 \moon[scale=\moonsize, sky colour=\skycolour]{1.206908}{Fri} &	 \moon[scale=\moonsize, sky colour=\skycolour]{2.450565}{Mon} &	 \moon[scale=\moonsize, sky colour=\skycolour]{2.722264}{Wed}	 & \numberDay{4}\\ 
	 \numberDay{5} & \moon[scale=\moonsize, sky colour=\skycolour]{23.029563}{Fri} &	 \moon[scale=\moonsize, sky colour=\skycolour]{24.560908}{Mon} &	 \moon[scale=\moonsize, sky colour=\skycolour]{24.134665}{Tue} &	 \moon[scale=\moonsize, sky colour=\skycolour]{25.748026}{Fri} &	 \moon[scale=\moonsize, sky colour=\skycolour]{26.373840}{Sun} &	 \moon[scale=\moonsize, sky colour=\skycolour]{27.996912}{Wed} &	 \moon[scale=\moonsize, sky colour=\skycolour]{28.570850}{Fri} &	 \moon[scale=\moonsize, sky colour=\skycolour]{0.531172}{Mon} &	 \moon[scale=\moonsize, sky colour=\skycolour]{1.908575}{Thu} &	 \moon[scale=\moonsize, sky colour=\skycolour]{2.199582}{Sat} &	 \moon[scale=\moonsize, sky colour=\skycolour]{3.443786}{Tue} &	 \moon[scale=\moonsize, sky colour=\skycolour]{3.717553}{Thu}	 & \numberDay{5}\\ 
	 \numberDay{6} & \moon[scale=\moonsize, sky colour=\skycolour]{24.030003}{Sat} &	 \moon[scale=\moonsize, sky colour=\skycolour]{25.563319}{Tue} &	 \moon[scale=\moonsize, sky colour=\skycolour]{25.138508}{Wed} &	 \moon[scale=\moonsize, sky colour=\skycolour]{26.752837}{Sat} &	 \moon[scale=\moonsize, sky colour=\skycolour]{27.379111}{Mon} &	 \moon[scale=\moonsize, sky colour=\skycolour]{29.001834}{Thu} &	 \moon[scale=\moonsize, sky colour=\skycolour]{0.043549}{Sat} &	 \moon[scale=\moonsize, sky colour=\skycolour]{1.528393}{Tue} &	 \moon[scale=\moonsize, sky colour=\skycolour]{2.902738}{Fri} &	 \moon[scale=\moonsize, sky colour=\skycolour]{3.192256}{Sun} &	 \moon[scale=\moonsize, sky colour=\skycolour]{4.437007}{Wed} &	 \moon[scale=\moonsize, sky colour=\skycolour]{4.712841}{Fri}	 & \numberDay{6}\\ 
	 \numberDay{7} & \moon[scale=\moonsize, sky colour=\skycolour]{25.030442}{Sun} &	 \moon[scale=\moonsize, sky colour=\skycolour]{26.565730}{Wed} &	 \moon[scale=\moonsize, sky colour=\skycolour]{26.142351}{Thu} &	 \moon[scale=\moonsize, sky colour=\skycolour]{27.757649}{Sun} &	 \moon[scale=\moonsize, sky colour=\skycolour]{28.384382}{Tue} &	 \moon[scale=\moonsize, sky colour=\skycolour]{0.475449}{Fri} &	 \moon[scale=\moonsize, sky colour=\skycolour]{1.044218}{Sun} &	 \moon[scale=\moonsize, sky colour=\skycolour]{2.525615}{Wed} &	 \moon[scale=\moonsize, sky colour=\skycolour]{3.896902}{Sat} &	 \moon[scale=\moonsize, sky colour=\skycolour]{4.184931}{Mon} &	 \moon[scale=\moonsize, sky colour=\skycolour]{5.430228}{Thu} &	 \moon[scale=\moonsize, sky colour=\skycolour]{5.708129}{Sat}	 & \numberDay{7}\\ 
	 \numberDay{8} & \moon[scale=\moonsize, sky colour=\skycolour]{26.030882}{Mon} &	 \moon[scale=\moonsize, sky colour=\skycolour]{27.568141}{Thu} &	 \moon[scale=\moonsize, sky colour=\skycolour]{27.146194}{Fri} &	 \moon[scale=\moonsize, sky colour=\skycolour]{28.762460}{Mon} &	 \moon[scale=\moonsize, sky colour=\skycolour]{29.389653}{Wed} &	 \moon[scale=\moonsize, sky colour=\skycolour]{1.478856}{Sat} &	 \moon[scale=\moonsize, sky colour=\skycolour]{2.044886}{Mon} &	 \moon[scale=\moonsize, sky colour=\skycolour]{3.522836}{Thu} &	 \moon[scale=\moonsize, sky colour=\skycolour]{4.891065}{Sun} &	 \moon[scale=\moonsize, sky colour=\skycolour]{5.177605}{Tue} &	 \moon[scale=\moonsize, sky colour=\skycolour]{6.423449}{Fri} &	 \moon[scale=\moonsize, sky colour=\skycolour]{6.703417}{Sun}	 & \numberDay{8}\\ 
	 \numberDay{9} & \moon[scale=\moonsize, sky colour=\skycolour]{27.031321}{Tue} &	 \moon[scale=\moonsize, sky colour=\skycolour]{28.570552}{Fri} &	 \moon[scale=\moonsize, sky colour=\skycolour]{28.150037}{Sat} &	 \moon[scale=\moonsize, sky colour=\skycolour]{0.236791}{Tue} &	 \moon[scale=\moonsize, sky colour=\skycolour]{0.864035}{Thu} &	 \moon[scale=\moonsize, sky colour=\skycolour]{2.482263}{Sun} &	 \moon[scale=\moonsize, sky colour=\skycolour]{3.045554}{Tue} &	 \moon[scale=\moonsize, sky colour=\skycolour]{4.520057}{Fri} &	 \moon[scale=\moonsize, sky colour=\skycolour]{5.885228}{Mon} &	 \moon[scale=\moonsize, sky colour=\skycolour]{6.170279}{Wed} &	 \moon[scale=\moonsize, sky colour=\skycolour]{7.416670}{Sat} &	 \moon[scale=\moonsize, sky colour=\skycolour]{7.698705}{Mon}	 & \numberDay{9}\\ 
	 \numberDay{10} & \moon[scale=\moonsize, sky colour=\skycolour]{28.031761}{Wed} &	 \moon[scale=\moonsize, sky colour=\skycolour]{0.042435}{Sat} &	 \moon[scale=\moonsize, sky colour=\skycolour]{29.153880}{Sun} &	 \moon[scale=\moonsize, sky colour=\skycolour]{1.242062}{Wed} &	 \moon[scale=\moonsize, sky colour=\skycolour]{1.868956}{Fri} &	 \moon[scale=\moonsize, sky colour=\skycolour]{3.485670}{Mon} &	 \moon[scale=\moonsize, sky colour=\skycolour]{4.046222}{Wed} &	 \moon[scale=\moonsize, sky colour=\skycolour]{5.517279}{Sat} &	 \moon[scale=\moonsize, sky colour=\skycolour]{6.879391}{Tue} &	 \moon[scale=\moonsize, sky colour=\skycolour]{7.162953}{Thu} &	 \moon[scale=\moonsize, sky colour=\skycolour]{8.409892}{Sun} &	 \moon[scale=\moonsize, sky colour=\skycolour]{8.693993}{Tue}	 & \numberDay{10}\\ 
	 \numberDay{11} & \moon[scale=\moonsize, sky colour=\skycolour]{29.032201}{Thu} &	 \moon[scale=\moonsize, sky colour=\skycolour]{1.046278}{Sun} &	 \moon[scale=\moonsize, sky colour=\skycolour]{0.627739}{Mon} &	 \moon[scale=\moonsize, sky colour=\skycolour]{2.247333}{Thu} &	 \moon[scale=\moonsize, sky colour=\skycolour]{2.873878}{Sat} &	 \moon[scale=\moonsize, sky colour=\skycolour]{4.489077}{Tue} &	 \moon[scale=\moonsize, sky colour=\skycolour]{5.046891}{Thu} &	 \moon[scale=\moonsize, sky colour=\skycolour]{6.514500}{Sun} &	 \moon[scale=\moonsize, sky colour=\skycolour]{7.873554}{Wed} &	 \moon[scale=\moonsize, sky colour=\skycolour]{8.155627}{Fri} &	 \moon[scale=\moonsize, sky colour=\skycolour]{9.403113}{Mon} &	 \moon[scale=\moonsize, sky colour=\skycolour]{9.689282}{Wed}	 & \numberDay{11}\\ 
	 \numberDay{12} & \moon[scale=\moonsize, sky colour=\skycolour]{0.503041}{Fri} &	 \moon[scale=\moonsize, sky colour=\skycolour]{2.050121}{Mon} &	 \moon[scale=\moonsize, sky colour=\skycolour]{1.632551}{Tue} &	 \moon[scale=\moonsize, sky colour=\skycolour]{3.252604}{Fri} &	 \moon[scale=\moonsize, sky colour=\skycolour]{3.878799}{Sun} &	 \moon[scale=\moonsize, sky colour=\skycolour]{5.492485}{Wed} &	 \moon[scale=\moonsize, sky colour=\skycolour]{6.047559}{Fri} &	 \moon[scale=\moonsize, sky colour=\skycolour]{7.511722}{Mon} &	 \moon[scale=\moonsize, sky colour=\skycolour]{8.867718}{Thu} &	 \moon[scale=\moonsize, sky colour=\skycolour]{9.148301}{Sat} &	 \moon[scale=\moonsize, sky colour=\skycolour]{10.396334}{Tue} &	 \moon[scale=\moonsize, sky colour=\skycolour]{10.684570}{Thu}	 & \numberDay{12}\\ 
	 \numberDay{13} & \moon[scale=\moonsize, sky colour=\skycolour]{1.505452}{Sat} &	 \moon[scale=\moonsize, sky colour=\skycolour]{3.053964}{Tue} &	 \moon[scale=\moonsize, sky colour=\skycolour]{2.637362}{Wed} &	 \moon[scale=\moonsize, sky colour=\skycolour]{4.257876}{Sat} &	 \moon[scale=\moonsize, sky colour=\skycolour]{4.883720}{Mon} &	 \moon[scale=\moonsize, sky colour=\skycolour]{6.495892}{Thu} &	 \moon[scale=\moonsize, sky colour=\skycolour]{7.048227}{Sat} &	 \moon[scale=\moonsize, sky colour=\skycolour]{8.508943}{Tue} &	 \moon[scale=\moonsize, sky colour=\skycolour]{9.861881}{Fri} &	 \moon[scale=\moonsize, sky colour=\skycolour]{10.140975}{Sun} &	 \moon[scale=\moonsize, sky colour=\skycolour]{11.389555}{Wed} &	 \moon[scale=\moonsize, sky colour=\skycolour]{11.679858}{Fri}	 & \numberDay{13}\\ 
	 \numberDay{14} & \moon[scale=\moonsize, sky colour=\skycolour]{2.507863}{Sun} &	 \moon[scale=\moonsize, sky colour=\skycolour]{4.057807}{Wed} &	 \moon[scale=\moonsize, sky colour=\skycolour]{3.642174}{Thu} &	 \moon[scale=\moonsize, sky colour=\skycolour]{5.263147}{Sun} &	 \moon[scale=\moonsize, sky colour=\skycolour]{5.888642}{Tue} &	 \moon[scale=\moonsize, sky colour=\skycolour]{7.499299}{Fri} &	 \moon[scale=\moonsize, sky colour=\skycolour]{8.048895}{Sun} &	 \moon[scale=\moonsize, sky colour=\skycolour]{9.506164}{Wed} &	 \moon[scale=\moonsize, sky colour=\skycolour]{10.856044}{Sat} &	 \moon[scale=\moonsize, sky colour=\skycolour]{11.133649}{Mon} &	 \moon[scale=\moonsize, sky colour=\skycolour]{12.382776}{Thu} &	 \moon[scale=\moonsize, sky colour=\skycolour]{12.675146}{Sat}	 & \numberDay{14}\\ 
	 \numberDay{15} & \moon[scale=\moonsize, sky colour=\skycolour]{3.510274}{Mon} &	 \moon[scale=\moonsize, sky colour=\skycolour]{5.061650}{Thu} &	 \moon[scale=\moonsize, sky colour=\skycolour]{4.646985}{Fri} &	 \moon[scale=\moonsize, sky colour=\skycolour]{6.268418}{Mon} &	 \moon[scale=\moonsize, sky colour=\skycolour]{6.893563}{Wed} &	 \moon[scale=\moonsize, sky colour=\skycolour]{8.502706}{Sat} &	 \moon[scale=\moonsize, sky colour=\skycolour]{9.049564}{Mon} &	 \moon[scale=\moonsize, sky colour=\skycolour]{10.503386}{Thu} &	 \moon[scale=\moonsize, sky colour=\skycolour]{11.850207}{Sun} &	 \moon[scale=\moonsize, sky colour=\skycolour]{12.126323}{Tue} &	 \moon[scale=\moonsize, sky colour=\skycolour]{13.375997}{Fri} &	 \moon[scale=\moonsize, sky colour=\skycolour]{13.670434}{Sun}	 & \numberDay{15}\\ 
	 \numberDay{16} & \moon[scale=\moonsize, sky colour=\skycolour]{4.512685}{Tue} &	 \moon[scale=\moonsize, sky colour=\skycolour]{6.065493}{Fri} &	 \moon[scale=\moonsize, sky colour=\skycolour]{5.651797}{Sat} &	 \moon[scale=\moonsize, sky colour=\skycolour]{7.273689}{Tue} &	 \moon[scale=\moonsize, sky colour=\skycolour]{7.898484}{Thu} &	 \moon[scale=\moonsize, sky colour=\skycolour]{9.506113}{Sun} &	 \moon[scale=\moonsize, sky colour=\skycolour]{10.050232}{Tue} &	 \moon[scale=\moonsize, sky colour=\skycolour]{11.500607}{Fri} &	 \moon[scale=\moonsize, sky colour=\skycolour]{12.844370}{Mon} &	 \moon[scale=\moonsize, sky colour=\skycolour]{13.118997}{Wed} &	 \moon[scale=\moonsize, sky colour=\skycolour]{14.369219}{Sat} &	 \moon[scale=\moonsize, sky colour=\skycolour]{14.665722}{Mon}	 & \numberDay{16}\\ 
	 \numberDay{17} & \moon[scale=\moonsize, sky colour=\skycolour]{5.515096}{Wed} &	 \moon[scale=\moonsize, sky colour=\skycolour]{7.069336}{Sat} &	 \moon[scale=\moonsize, sky colour=\skycolour]{6.656608}{Sun} &	 \moon[scale=\moonsize, sky colour=\skycolour]{8.278960}{Wed} &	 \moon[scale=\moonsize, sky colour=\skycolour]{8.903406}{Fri} &	 \moon[scale=\moonsize, sky colour=\skycolour]{10.509521}{Mon} &	 \moon[scale=\moonsize, sky colour=\skycolour]{11.050900}{Wed} &	 \moon[scale=\moonsize, sky colour=\skycolour]{12.497829}{Sat} &	 \moon[scale=\moonsize, sky colour=\skycolour]{13.838534}{Tue} &	 \moon[scale=\moonsize, sky colour=\skycolour]{14.111671}{Thu} &	 \moon[scale=\moonsize, sky colour=\skycolour]{15.362440}{Sun} &	 \moon[scale=\moonsize, sky colour=\skycolour]{15.661010}{Tue}	 & \numberDay{17}\\ 
	 \numberDay{18} & \moon[scale=\moonsize, sky colour=\skycolour]{6.517507}{Thu} &	 \moon[scale=\moonsize, sky colour=\skycolour]{8.073179}{Sun} &	 \moon[scale=\moonsize, sky colour=\skycolour]{7.661419}{Mon} &	 \moon[scale=\moonsize, sky colour=\skycolour]{9.284231}{Thu} &	 \moon[scale=\moonsize, sky colour=\skycolour]{9.908327}{Sat} &	 \moon[scale=\moonsize, sky colour=\skycolour]{11.512928}{Tue} &	 \moon[scale=\moonsize, sky colour=\skycolour]{12.051568}{Thu} &	 \moon[scale=\moonsize, sky colour=\skycolour]{13.495050}{Sun} &	 \moon[scale=\moonsize, sky colour=\skycolour]{14.832697}{Wed} &	 \moon[scale=\moonsize, sky colour=\skycolour]{15.104345}{Fri} &	 \moon[scale=\moonsize, sky colour=\skycolour]{16.355661}{Mon} &	 \moon[scale=\moonsize, sky colour=\skycolour]{16.656299}{Wed}	 & \numberDay{18}\\ 
	 \numberDay{19} & \moon[scale=\moonsize, sky colour=\skycolour]{7.519919}{Fri} &	 \moon[scale=\moonsize, sky colour=\skycolour]{9.077021}{Mon} &	 \moon[scale=\moonsize, sky colour=\skycolour]{8.666231}{Tue} &	 \moon[scale=\moonsize, sky colour=\skycolour]{10.289502}{Fri} &	 \moon[scale=\moonsize, sky colour=\skycolour]{10.913249}{Sun} &	 \moon[scale=\moonsize, sky colour=\skycolour]{12.516335}{Wed} &	 \moon[scale=\moonsize, sky colour=\skycolour]{13.052237}{Fri} &	 \moon[scale=\moonsize, sky colour=\skycolour]{14.492271}{Mon} &	 \moon[scale=\moonsize, sky colour=\skycolour]{15.826860}{Thu} &	 \moon[scale=\moonsize, sky colour=\skycolour]{16.097019}{Sat} &	 \moon[scale=\moonsize, sky colour=\skycolour]{17.348882}{Tue} &	 \moon[scale=\moonsize, sky colour=\skycolour]{17.651587}{Thu}	 & \numberDay{19}\\ 
	 \numberDay{20} & \moon[scale=\moonsize, sky colour=\skycolour]{8.522330}{Sat} &	 \moon[scale=\moonsize, sky colour=\skycolour]{10.080864}{Tue} &	 \moon[scale=\moonsize, sky colour=\skycolour]{9.671042}{Wed} &	 \moon[scale=\moonsize, sky colour=\skycolour]{11.294773}{Sat} &	 \moon[scale=\moonsize, sky colour=\skycolour]{11.918170}{Mon} &	 \moon[scale=\moonsize, sky colour=\skycolour]{13.519742}{Thu} &	 \moon[scale=\moonsize, sky colour=\skycolour]{14.052905}{Sat} &	 \moon[scale=\moonsize, sky colour=\skycolour]{15.489493}{Tue} &	 \moon[scale=\moonsize, sky colour=\skycolour]{16.821023}{Fri} &	 \moon[scale=\moonsize, sky colour=\skycolour]{17.089693}{Sun} &	 \moon[scale=\moonsize, sky colour=\skycolour]{18.342103}{Wed} &	 \moon[scale=\moonsize, sky colour=\skycolour]{18.646875}{Fri}	 & \numberDay{20}\\ 
	 \numberDay{21} & \moon[scale=\moonsize, sky colour=\skycolour]{9.524741}{Sun} &	 \moon[scale=\moonsize, sky colour=\skycolour]{11.084707}{Wed} &	 \moon[scale=\moonsize, sky colour=\skycolour]{10.675854}{Thu} &	 \moon[scale=\moonsize, sky colour=\skycolour]{12.300044}{Sun} &	 \moon[scale=\moonsize, sky colour=\skycolour]{12.923091}{Tue} &	 \moon[scale=\moonsize, sky colour=\skycolour]{14.523149}{Fri} &	 \moon[scale=\moonsize, sky colour=\skycolour]{15.053573}{Sun} &	 \moon[scale=\moonsize, sky colour=\skycolour]{16.486714}{Wed} &	 \moon[scale=\moonsize, sky colour=\skycolour]{17.815186}{Sat} &	 \moon[scale=\moonsize, sky colour=\skycolour]{18.082367}{Mon} &	 \moon[scale=\moonsize, sky colour=\skycolour]{19.335325}{Thu} &	 \moon[scale=\moonsize, sky colour=\skycolour]{19.642163}{Sat}	 & \numberDay{21}\\ 
	 \numberDay{22} & \moon[scale=\moonsize, sky colour=\skycolour]{10.527152}{Mon} &	 \moon[scale=\moonsize, sky colour=\skycolour]{12.088550}{Thu} &	 \moon[scale=\moonsize, sky colour=\skycolour]{11.680665}{Fri} &	 \moon[scale=\moonsize, sky colour=\skycolour]{13.305316}{Mon} &	 \moon[scale=\moonsize, sky colour=\skycolour]{13.928013}{Wed} &	 \moon[scale=\moonsize, sky colour=\skycolour]{15.526557}{Sat} &	 \moon[scale=\moonsize, sky colour=\skycolour]{16.054241}{Mon} &	 \moon[scale=\moonsize, sky colour=\skycolour]{17.483936}{Thu} &	 \moon[scale=\moonsize, sky colour=\skycolour]{18.809349}{Sun} &	 \moon[scale=\moonsize, sky colour=\skycolour]{19.075041}{Tue} &	 \moon[scale=\moonsize, sky colour=\skycolour]{20.328546}{Fri} &	 \moon[scale=\moonsize, sky colour=\skycolour]{20.637451}{Sun}	 & \numberDay{22}\\ 
	 \numberDay{23} & \moon[scale=\moonsize, sky colour=\skycolour]{11.529563}{Tue} &	 \moon[scale=\moonsize, sky colour=\skycolour]{13.092393}{Fri} &	 \moon[scale=\moonsize, sky colour=\skycolour]{12.685477}{Sat} &	 \moon[scale=\moonsize, sky colour=\skycolour]{14.310587}{Tue} &	 \moon[scale=\moonsize, sky colour=\skycolour]{14.932934}{Thu} &	 \moon[scale=\moonsize, sky colour=\skycolour]{16.529964}{Sun} &	 \moon[scale=\moonsize, sky colour=\skycolour]{17.054910}{Tue} &	 \moon[scale=\moonsize, sky colour=\skycolour]{18.481157}{Fri} &	 \moon[scale=\moonsize, sky colour=\skycolour]{19.803513}{Mon} &	 \moon[scale=\moonsize, sky colour=\skycolour]{20.067715}{Wed} &	 \moon[scale=\moonsize, sky colour=\skycolour]{21.321767}{Sat} &	 \moon[scale=\moonsize, sky colour=\skycolour]{21.632739}{Mon}	 & \numberDay{23}\\ 
	 \numberDay{24} & \moon[scale=\moonsize, sky colour=\skycolour]{12.531974}{Wed} &	 \moon[scale=\moonsize, sky colour=\skycolour]{14.096236}{Sat} &	 \moon[scale=\moonsize, sky colour=\skycolour]{13.690288}{Sun} &	 \moon[scale=\moonsize, sky colour=\skycolour]{15.315858}{Wed} &	 \moon[scale=\moonsize, sky colour=\skycolour]{15.937856}{Fri} &	 \moon[scale=\moonsize, sky colour=\skycolour]{17.533371}{Mon} &	 \moon[scale=\moonsize, sky colour=\skycolour]{18.055578}{Wed} &	 \moon[scale=\moonsize, sky colour=\skycolour]{19.478378}{Sat} &	 \moon[scale=\moonsize, sky colour=\skycolour]{20.797676}{Tue} &	 \moon[scale=\moonsize, sky colour=\skycolour]{21.060389}{Thu} &	 \moon[scale=\moonsize, sky colour=\skycolour]{22.314988}{Sun} &	 \moon[scale=\moonsize, sky colour=\skycolour]{22.628028}{Tue}	 & \numberDay{24}\\ 
	 \numberDay{25} & \moon[scale=\moonsize, sky colour=\skycolour]{13.534385}{Thu} &	 \moon[scale=\moonsize, sky colour=\skycolour]{15.100079}{Sun} &	 \moon[scale=\moonsize, sky colour=\skycolour]{14.695100}{Mon} &	 \moon[scale=\moonsize, sky colour=\skycolour]{16.321129}{Thu} &	 \moon[scale=\moonsize, sky colour=\skycolour]{16.942777}{Sat} &	 \moon[scale=\moonsize, sky colour=\skycolour]{18.536778}{Tue} &	 \moon[scale=\moonsize, sky colour=\skycolour]{19.056246}{Thu} &	 \moon[scale=\moonsize, sky colour=\skycolour]{20.475600}{Sun} &	 \moon[scale=\moonsize, sky colour=\skycolour]{21.791839}{Wed} &	 \moon[scale=\moonsize, sky colour=\skycolour]{22.053063}{Fri} &	 \moon[scale=\moonsize, sky colour=\skycolour]{23.308209}{Mon} &	 \moon[scale=\moonsize, sky colour=\skycolour]{23.623316}{Wed}	 & \numberDay{25}\\ 
	 \numberDay{26} & \moon[scale=\moonsize, sky colour=\skycolour]{14.536796}{Fri} &	 \moon[scale=\moonsize, sky colour=\skycolour]{16.103922}{Mon} &	 \moon[scale=\moonsize, sky colour=\skycolour]{15.699911}{Tue} &	 \moon[scale=\moonsize, sky colour=\skycolour]{17.326400}{Fri} &	 \moon[scale=\moonsize, sky colour=\skycolour]{17.947698}{Sun} &	 \moon[scale=\moonsize, sky colour=\skycolour]{19.540185}{Wed} &	 \moon[scale=\moonsize, sky colour=\skycolour]{20.056914}{Fri} &	 \moon[scale=\moonsize, sky colour=\skycolour]{21.472821}{Mon} &	 \moon[scale=\moonsize, sky colour=\skycolour]{22.786002}{Thu} &	 \moon[scale=\moonsize, sky colour=\skycolour]{23.045737}{Sat} &	 \moon[scale=\moonsize, sky colour=\skycolour]{24.301430}{Tue} &	 \moon[scale=\moonsize, sky colour=\skycolour]{24.618604}{Thu}	 & \numberDay{26}\\ 
	 \numberDay{27} & \moon[scale=\moonsize, sky colour=\skycolour]{15.539208}{Sat} &	 \moon[scale=\moonsize, sky colour=\skycolour]{17.107765}{Tue} &	 \moon[scale=\moonsize, sky colour=\skycolour]{16.704723}{Wed} &	 \moon[scale=\moonsize, sky colour=\skycolour]{18.331671}{Sat} &	 \moon[scale=\moonsize, sky colour=\skycolour]{18.952620}{Mon} &	 \moon[scale=\moonsize, sky colour=\skycolour]{20.543593}{Thu} &	 \moon[scale=\moonsize, sky colour=\skycolour]{21.057582}{Sat} &	 \moon[scale=\moonsize, sky colour=\skycolour]{22.470043}{Tue} &	 \moon[scale=\moonsize, sky colour=\skycolour]{23.780165}{Fri} &	 \moon[scale=\moonsize, sky colour=\skycolour]{24.038411}{Sun} &	 \moon[scale=\moonsize, sky colour=\skycolour]{25.294652}{Wed} &	 \moon[scale=\moonsize, sky colour=\skycolour]{25.613892}{Fri}	 & \numberDay{27}\\ 
	 \numberDay{28} & \moon[scale=\moonsize, sky colour=\skycolour]{16.541619}{Sun} &	 \moon[scale=\moonsize, sky colour=\skycolour]{18.111608}{Wed} &	 \moon[scale=\moonsize, sky colour=\skycolour]{17.709534}{Thu} &	 \moon[scale=\moonsize, sky colour=\skycolour]{19.336942}{Sun} &	 \moon[scale=\moonsize, sky colour=\skycolour]{19.957541}{Tue} &	 \moon[scale=\moonsize, sky colour=\skycolour]{21.547000}{Fri} &	 \moon[scale=\moonsize, sky colour=\skycolour]{22.058251}{Sun} &	 \moon[scale=\moonsize, sky colour=\skycolour]{23.467264}{Wed} &	 \moon[scale=\moonsize, sky colour=\skycolour]{24.774329}{Sat} &	 \moon[scale=\moonsize, sky colour=\skycolour]{25.031085}{Mon} &	 \moon[scale=\moonsize, sky colour=\skycolour]{26.287873}{Thu} &	 \moon[scale=\moonsize, sky colour=\skycolour]{26.609180}{Sat}	 & \numberDay{28}\\ 
	 \numberDay{29} & \moon[scale=\moonsize, sky colour=\skycolour]{17.544030}{Mon} &	 \moon[scale=\moonsize, sky colour=\skycolour]{19.115451}{Thu} &	 \moon[scale=\moonsize, sky colour=\skycolour]{18.714346}{Fri} &	 \moon[scale=\moonsize, sky colour=\skycolour]{20.342213}{Mon} &	 \moon[scale=\moonsize, sky colour=\skycolour]{20.962462}{Wed} &	 \moon[scale=\moonsize, sky colour=\skycolour]{22.550407}{Sat} &	 \moon[scale=\moonsize, sky colour=\skycolour]{23.058919}{Mon} &	 \moon[scale=\moonsize, sky colour=\skycolour]{24.464485}{Thu} &	 \moon[scale=\moonsize, sky colour=\skycolour]{25.768492}{Sun} &	 \moon[scale=\moonsize, sky colour=\skycolour]{26.023759}{Tue} &	 \moon[scale=\moonsize, sky colour=\skycolour]{27.281094}{Fri} &	 \moon[scale=\moonsize, sky colour=\skycolour]{27.604468}{Sun}	 & \numberDay{29}\\ 
	 \numberDay{30} & \moon[scale=\moonsize, sky colour=\skycolour]{18.546441}{Tue} &	& 	 \moon[scale=\moonsize, sky colour=\skycolour]{19.719157}{Sat} &	 \moon[scale=\moonsize, sky colour=\skycolour]{21.347484}{Tue} &	 \moon[scale=\moonsize, sky colour=\skycolour]{21.967384}{Thu} &	 \moon[scale=\moonsize, sky colour=\skycolour]{23.553814}{Sun} &	 \moon[scale=\moonsize, sky colour=\skycolour]{24.059587}{Tue} &	 \moon[scale=\moonsize, sky colour=\skycolour]{25.461707}{Fri} &	 \moon[scale=\moonsize, sky colour=\skycolour]{26.762655}{Mon} &	 \moon[scale=\moonsize, sky colour=\skycolour]{27.016433}{Wed} &	 \moon[scale=\moonsize, sky colour=\skycolour]{28.274315}{Sat} &	 \moon[scale=\moonsize, sky colour=\skycolour]{28.599756}{Mon}	 & \numberDay{30}\\ 
	 \numberDay{31} & \moon[scale=\moonsize, sky colour=\skycolour]{19.548852}{Wed} &	& 	 \moon[scale=\moonsize, sky colour=\skycolour]{20.723969}{Sun} &	& 	 \moon[scale=\moonsize, sky colour=\skycolour]{22.972305}{Fri} &	& 	 \moon[scale=\moonsize, sky colour=\skycolour]{25.060255}{Wed} &	 \moon[scale=\moonsize, sky colour=\skycolour]{26.458928}{Sat} &	& 	 \moon[scale=\moonsize, sky colour=\skycolour]{28.009107}{Thu} &	& 	 \moon[scale=\moonsize, sky colour=\skycolour]{0.064632}{Tue}	 & \numberDay{31}\\ 
\end{tabular} 
 \vspace{2em}

\end{tabular}

\end{document}